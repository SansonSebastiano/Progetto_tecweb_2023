\section{Progettazione}
\label{sec:progettazione}
\subsection{Struttura del progetto}
\subsubsection{Struttura delle cartelle}
Il progetto segue la seguente sotto-struttura di cartelle:
\begin{itemize}
	\item \textbf{html}: cartella contenente la struttura delle varie pagine del sito e una cartella \textit{modules} contenente alcuni moduli che vanno caricati all'interno di tutte le pagine, come header e footer, o solo in alcune specifiche, come per i commenti.
	\item \textbf{css}: cartella contenente tutti i file che definiscono l'aspetto delle pagine del sito, inoltre sono presenti i css per la visualizzazione mobile e per la stampa;
	\item \textbf{php}: questa cartella raccoglie gli script necessari per l'esecuzione di query di scrittura nel DB e una sottocartella \textit{pages} che contiene i file per rendere dinamiche le pagine del sito;
	\item \textbf{js}: cartella contenente i file javascript, il quale ognuno si occupa di implementare una specifica funzionalità nel sito;
	\item \textbf{images}: cartella contenente icone, loghi o sfondi presenti nel sito; 
	\item \textbf{config.php}: file di configurazione dei vari percorsi utilizzati nel sito;
	\item \textbf{index.php}: file dell'homepage del sito.
\end{itemize}
\subsubsection{Approccio allo sviluppo}
Per lo sviluppo del sito si sono voluti separare al meglio struttura, aspetto e comportamento. Ogni pagina html ha infatti un suo corrispettivo foglio di stile e uno script php associato, il quale serve ad importare la struttura della pagina e renderla dinamica. Eventualmente le pagine che fanno utilizzo di particolari funzionalità importano al loro interno i file javascript nelle quali sono scritte le funzioni che permettono di svolgere tali operazioni.\\
Vi sono inoltre altri script php, non visibili all'utente, che sono utilizzate per creare connessioni con il database e svolgere le query di scrittura su esso. Questo tipo di pagine servono anche per lo svolgimento dei controlli back-end di validazione dell'input.

Durante lo sviluppo 

\subsection{Database}
Il database è composto da un totale di 6 entità:
\begin{itemize}
	\item \textbf{Animale}: che contiene le informazioni di tutte le creature (scoperte, avvistate o ipotizzate) mostrate nel sito;
	\item \textbf{Articolo}: che contiene le informazioni degli articoli pubblicati. Un articolo può riferire un animale ma non necessariamente;
	\item \textbf{Utente}: che contiene i dati dei vari utenti registrati al sito, compreso il loro ruolo;
	\item \textbf{Commento}: che contiene tutti i vari commenti presenti sotto agli articoli, con le informazioni sul dove si trovino e chi li abbia scritti;
	\item \textbf{Risposta}: Entità che modella la relazione tra commento padre e figli;
	\item \textbf{Voto}: Entità che modella  i singoli voti degli utenti per ogni animale.
\end{itemize}
Sono inoltre state aggiunte delle view al database per semplificare alcune query che vengono spesso eseguite all'interno del sito:
\begin{itemize}
	\item \textbf{View\_animale\_voto}: Che correla ogni animale alla quantità di voti ricevuti;
	\item \textbf{View\_articolo\_commento}: Che correla ogni articolo alla lista di commenti padre presenti in esso;
	\item \textbf{View\_articolo\_commento\_risposta}: Che correla ogni commento padre di uno specifico articolo ai propri figli;
	\item \textbf{View\_articolo\_utente}: Che correla ogni articolo al nome dell'utente che lo ha scritto;
	\item \textbf{Vote\_no}: Che raccoglie il numero di voti a sfavore dell'esistenza di un certo animale;
	\item \textbf{Vote\_yes}: Che raccoglie il numero di voti a favore dell'esistenza di un certo animale;
\end{itemize}

