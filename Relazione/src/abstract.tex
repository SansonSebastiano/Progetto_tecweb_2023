\section{Abstract}
\label{sec:abstract}
Il sito si propone come un aggregatore di notizie riguardanti il mondo della criptozoologia, creature, pseudoscienza il cui ambito di interesse ricade su animali leggendari la cui esistenza è controversa o non provata. \\
Questa piattaforma permette di usufruire delle seguenti funzionalità:
\begin{itemize}
    \item \textbf{Login e Logout}: gli utenti possono accedere al sito tramite un form di login, che permette di accedere alle funzionalità riservate agli utenti che possiedono un account. Inoltre, è possibile effettuare il logout, che termina la sessione corrente.
    \item \textbf{Lettura}, riservata a qualunque utente, anche non possessore di un account, permette di visualizzare i seguenti contenuti del sito:
    \begin{itemize}
        \item \textbf{Articoli}: possibilità consultare gli articoli presenti nel sito, il contenuto di ciascuno si può riferire ad una creatura in particolare;
        \item \textbf{Creature}: possibilità di visualizzare le caratteristiche delle creature presenti nel sito;
        \item \textbf{Lista degli articoli}: possibilità di effettuare una ricerca attraverso alcuni filtri;
        \item \textbf{Lista delle creature}: elencate in ordine alfabetico;
        \item \textbf{Classifica delle creature}: possibilità di visualizzare le creature ordinate secondo tre differenti criteri, che verrano descritti successivamente;
    \end{itemize}
    \item \textbf{Scrittura}, riservata agli utenti che possiedono un account, permette di usufruire, in aggiunta, delle seguenti funzionalità, inoltre alcune tra queste sono riservate agli utenti con ruolo di autore o amministratore (verranno opportunamente segnalate):
    \begin{itemize}
        \item \textbf{Votazione}: per ciascuna creatura è possibile esprimere un voto riguardante la sua esistenza, inoltre vi è la possibilità di retrocedere il voto precedentemente espresso, a causa di un errore di valutazione;
        \item \textbf{Commenti}: per ciascun articolo vi è la possibilità di commentare il suo contenuto;
        \item Di seguito verranno elencate le funzionalità riservate agli utenti con ruolo di autore e amministratore:
        \begin{itemize}
            \item \textbf{Creazione di un articolo}: possibilità di creare un nuovo articolo (riservata agli autori e agli amministratori);
            \item \textbf{Aggiunta di una creatura}: possibilità di aggiungere una nuova creatura (riservata agli amministratori);
            \item \textbf{Modifica di un articolo}: possibilità di modificare solamente il contenuto di un articolo (riservata agli amministratori);
            \item \textbf{Rimozione di un articolo}: possibilità di rimuovere un articolo (riservata agli amministratori);
            \item \textbf{Rimozione di una creatura}: possibilità di rimuovere una creatura (riservata agli amministratori);
        \end{itemize}
    \end{itemize}
\end{itemize}