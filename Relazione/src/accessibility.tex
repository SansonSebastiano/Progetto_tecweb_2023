\section{Accessibilità}
\label{sec:accessibility}

\subsection{Riferimenti normativi}
\label{subsec:accessibility-norms}

\begin{itemize}
    \item \textbf{WCAG4All}: \url{https://web.math.unipd.it/accessibility/test.html}
    \item \textbf{Standard WCAG 2.1}: \url{https://www.w3.org/Translations/WCAG21-it/}
\end{itemize}

\subsection{Strumenti utilizzati}
\begin{itemize}
    \item \textbf{WAVE}: \url{https://wave.webaim.org/}
    \item \textbf{Total validator}: \url{https://www.totalvalidator.com/}
    \item \textbf{Accessibility Insights for Web}: \url{https://accessibilityinsights.io/}
    \item \textbf{Siteimprove A.C.}: \url{https://www.siteimprove.com/toolkit/accessibility-checker/}
\end{itemize}

\subsection{Verifica}
\label{subsec:accessibility-verification}

Per garantire il livello di accessibilità richiesto (livello AA di WCAG4All), sono stati effettuati alcuni test elencati in \ref{subsec:accessibility-norms}, i quali verrano discussi di seguito, i test di livello AAA non sono stati presi in considerazione.

\subsubsection{HTML}
\label{subsubsec:accessibility-html}

Sono stati verificati, attraverso procedure manuali o automatiche, i seguenti requisiti:
\begin{itemize}
    \item verifica dell'uso corretto degli elementi semantici di HTML5 per la definizione della struttura delle pagine;
    \item verifica dell'uso corretto degli elementi semantici per dare enfasi al contenuto;
    \item verifica della lingua utilizzata per ogni pagina;
    \item verifica della presenza di ID duplicati sulla stessa pagina;
\end{itemize}

\subsubsection{CSS}
\label{subsubsec:accessibility-css}

Sono stati verificati, attraverso procedure manuali o automatiche, i seguenti requisiti:
\begin{itemize}
    \item verifica che le immagini a scopo decorativo siano stati applicate via CSS;
    \item verifica dell'implementazione di unità di misure relative;
    \item verifica della scalabilità del layout del sito, sono stati implementati fogli di stile necessari per la visualizzazione del sito su dispositivi con schermi di dimensioni ridotte;
\end{itemize}

Inoltre sono stati implementati fogli di stile per la stampa delle pagine e si è verificato che in assenza di fogli di stile il sito sia comunque navigabile.

\subsubsection{JavaScript}
\label{subsubsec:accessibility-js}

Attraverso \textit{JavaScript} sono stati implementati alcuni script per il comportamento del sito e per la validazione degli input lato client.

\subsubsection{Immagini}
\label{subsubsec:accessibility-images}

Sono state verificate le seguenti proprietà delle immagini:
\begin{itemize}
    \item le immagini puramente decorative sono state implementate via \textit{CSS};
    \item le immagini di contenuto invece sono state definite via \textit{HTML} e sono state fornite alternative testuali opportune;
\end{itemize}

\subsubsection{Link}
\label{subsubsec:accessibility-links}

\subsubsection{Colori}
\label{subsubsec:accessibility-colors}

\subsubsection{Form}
\label{subsubsec:accessibility-forms}

\subsubsection{Tabelle}
\label{subsubsec:accessibility-tables}

\subsubsection{Modalità di input}
\label{subsubsec:accessibility-input}

\subsubsection{Tastiera}
\label{subsubsec:accessibility-keyboard}

\subsubsection{Ruoli, stati e proprietà WAI ARIA}
\label{subsubsec:accessibility-aria}

Ruoli \textit{WAI ARIA} non sono stati necessari per la realizzazione del sito, in quanto sono risultati sufficienti gli elementi semantici di \textit{HTML5}. \\
Dunque non sono stati implementati in maniera sostanziale neanche gli stati e le proprietà \textit{WAI ARIA}, eccetto per alcuni casi che verranno discussi di seguito:
\begin{itemize}
    \item \textbf{Breadcrumb}: è stato implementato un \textit{aria-label} per specificarne il funzionamento;
    \item \textbf{Validazione input}: è stato implementato un \textit{aria-label} per esplicitare eventuali errori nell'inserimento degli input;
    \item \textbf{Pulsanti}: è stato implementato un \textit{aria-label} nei pulsanti \textit{Submit}, di rimozione e modifica di contenuti per esplicitarne lo scopo;
\end{itemize}

\subsubsection{Aiuti alla navigazione}
\label{subsubsec:accessibility-navigation}

\subsubsection{Separazione tra struttura, contenuto e comportamento}
\label{subsubsec:accessibility-separation}