\section{Accessibilità}
\label{sec:accessibility}

\subsection{Riferimenti normativi}
\label{subsec:accessibility-norms}

\begin{itemize}
    \item \textbf{WCAG4All}: \url{https://web.math.unipd.it/accessibility/test.html}
    \item \textbf{Standard WCAG 2.1}: \url{https://www.w3.org/Translations/WCAG21-it/}
\end{itemize}

\subsection{Strumenti utilizzati}
\begin{itemize}
    \item \textbf{WAVE}: \url{https://wave.webaim.org/}
    \item \textbf{Total validator}: \url{https://www.totalvalidator.com/}
    \item \textbf{Accessibility Insights for Web}: \url{https://accessibilityinsights.io/}
    \item \textbf{Siteimprove A.C.}: \url{https://www.siteimprove.com/toolkit/accessibility-checker/}
\end{itemize}

\subsection{Verifica}
\label{subsec:accessibility-verification}

Per garantire il livello di accessibilità richiesto, sono stati effettuati dei test opportuni elencati, i quali verrano elencati di seguito, si specifica che quasi la totalità dei test per garantire il livello AAA non sono stati presi in considerazione.

\subsubsection{HTML}
\label{subsubsec:accessibility-html}

Sono stati verificati, attraverso procedure manuali o automatiche, i seguenti requisiti:
\begin{itemize}
    \item uso corretto degli elementi semantici di HTML5 per la definizione della struttura delle pagine;
    \item uso corretto degli elementi semantici per dare enfasi al contenuto;
    \item lingua principale utilizzata per ogni pagina;
    \item verifica della presenza di ID duplicati sulla stessa pagina.
\end{itemize}

\subsubsection{CSS}
\label{subsubsec:accessibility-css}

Sono stati verificati, attraverso procedure manuali o automatiche, i seguenti requisiti:
\begin{itemize}
    \item le immagini a scopo decorativo siano stati applicate via CSS;
    \item implementazione di unità di misure relative;
    \item scalabilità del layout del sito, sono stati implementati fogli di stile necessari per la visualizzazione del sito su dispositivi con schermi di dimensioni ridotte.
\end{itemize}

Inoltre sono stati implementati fogli di stile per la stampa delle pagine e si è verificato che in assenza di fogli di stile il sito sia comunque navigabile.

\subsubsection{JavaScript}
\label{subsubsec:accessibility-js}

Attraverso \textit{JavaScript} sono stati implementati alcuni script per il comportamento del sito e per la validazione degli input lato client. \\
Inoltre, in seguito, si è verificato che non vi siano presenti cambi di contesto inaspettati.

\subsubsection{Immagini}
\label{subsubsec:accessibility-images}

Sono state verificate le seguenti proprietà delle immagini:
\begin{itemize}
    \item le immagini puramente decorative sono state implementate via \textit{CSS};
    \item le immagini di contenuto invece sono state definite via \textit{HTML} e sono state fornite alternative testuali.opportune.
\end{itemize}

\subsubsection{Link}
\label{subsubsec:accessibility-links}
Sono stati verificati, attraverso procedure manuali o automatiche, i seguenti requisiti:
\begin{itemize}
    \item il colore non deve essere l'unico mezzo per identificare i link presenti nelle pagine;
    \item ancora dei link opportuna per far comprendere lo scopo di ciascuno di essi;
    \item comprendere la differenza tra link già visitati e non ancora visitati.
\end{itemize}

\subsubsection{Colori}
\label{subsubsec:accessibility-colors}
Sono state rispettate tutte le norme per garantire il contrasto (3:1) tra i colori di testo e lo sfondo, si è verificato, inoltre, che non siano l'unico mezzo per trasmettere informazioni. \\
Tuttavia ci sono testi che non presentano il grado di contrasto richiesto, in quanto sono dotati di proprieta \texttt{text-shadow} che fornisce un effetto di ombreggiatura al testo, per cui non è stato necessario intervenire.

\subsubsection{Form}
\label{subsubsec:accessibility-forms}
Sono stati verificati, attraverso procedure manuali o automatiche, i seguenti requisiti:
\begin{itemize}
    \item comprensibilità della form, ovvero quali sono i dati richiesti e se opzionali od obbligatori;
    \item gli errori identificabili e comprensibili, anche attraverso l'uso di \texttt{aria-role="alert"} per notificare all'utente, via screed reader, eventuali errori di compilazione;
    \item presenza di attibuti di autocompletamento nei campi delle form;
    \item compilabilitaq delle form anche senza l'uso del mouse.
\end{itemize}

\subsubsection{Tabelle}
\label{subsubsec:accessibility-tables}
Sono stati verificati, attraverso procedure manuali o automatiche, i seguenti requisiti:
\begin{itemize}
    \item progettazione corretta della struttura delle tabelle;
    \item presenza della \textit{caption} per descrivere lo scopo della tabella;
    \item trasformazione elegante delle tabelle in caso di visualizzazione delle stesse in dimensioni ridotte.
\end{itemize}

\subsubsection{Modalità di input}
\label{subsubsec:accessibility-input}
Sono stati verificate che le operazioni possano essere svolte con un solo click per tutti i dispositivi di input supportati, ovvero mouse, tastiera e touch screen. \\
Inoltre è stata implementata la possibilità di retrocedere dal voto espresso per una creatura, in seguito ad un click accidentale oppure ad un errore di valutazione dell'utente votante.

\subsubsection{Tastiera}
\label{subsubsec:accessibility-keyboard}
Verifica della corretta implementazione dell'attributo \textit{tabindex} per la navigazione del sito attraverso la tastiera nell'ordine corretto, inoltre l'elemento attualmente selezionato viene evidenziato in maniera opportuna. \\
È possibile usufruire di tutte le funzionalità offerte, servendosi dei controlli interattivi via tastiera, spostando il focus con il tasto \textit{TAB} e selezionando l'elemento desiderato con il tasto \textit{INVIO}. \\
Infine si specifica che non sono state implementate scorciatoie da tastiera.

\subsubsection{Ruoli, stati e proprietà WAI ARIA}
\label{subsubsec:accessibility-aria}

Ruoli \textit{WAI ARIA} non sono stati necessari per la realizzazione del sito, in quanto sono risultati sufficienti gli elementi semantici di \textit{HTML5}. \\
Dunque non sono stati implementati in maniera sostanziale neanche gli stati e le proprietà \textit{WAI ARIA}, eccetto per alcuni casi che verranno discussi di seguito:
\begin{itemize}
    \item \textbf{Breadcrumb}: è stato implementato un \texttt{aria-label} per specificarne il funzionamento;
    \item \textbf{Validazione input}: è stato implementato un \texttt{aria-label} per esplicitare eventuali errori nell'inserimento degli input;
    \item \textbf{Pulsanti}: è stato implementato un \texttt{aria-label} nei pulsanti \textit{Submit}, di rimozione e modifica di contenuti per esplicitarne lo scopo.
\end{itemize}

\subsubsection{Aiuti alla navigazione}
\label{subsubsec:accessibility-navigation}
È stato implementato, nelle pagine del sito in cui è necessario uno \textit{scroll} per visualizzarne l'intero contenuto, un pulsante che permette di tornare in cima alla pagina.

\subsubsection{Disorientamento}
\label{subsubsec:accessibility-disorientation}
Per evitare questo fenomeno, si sono verificate le seguenti proprietà:
\begin{itemize}
    \item il titolo rispetta la convenzione per cui deve essere dal particolare al generale;
    \item gli elementi ricorrenti nel sito, sono sempre posizionati nello stesso modo;
    \item presenza di una breadcrumb per fornire l'utente un orientamento all'interno del sito;
    \item possibilità di tornare alla home page in qualsiasi momento, cliccando sul logo del sito;
    \item possibilità di visualizzare il sito in modalità desktop e mobile, grazie all'implementazione di fogli di stile opportuni.
\end{itemize}

\subsubsection{Separazione tra struttura, contenuto e comportamento}
\label{subsubsec:accessibility-separation}
Per garantire la separazione tra struttura, contenuto e comportamento, si è fatto uso di fogli di stile esterni, in modo da poter modificare l'aspetto del sito senza dover intervenire sul codice \textit{HTML}. \\
Analogamente, si sono impiegati script \textit{PHP} e \textit{JavaScript} esterni, rispettivamente per la gestione delle funzionalità lato server, templating delle pagine \textit{HTML} e per la gestione delle funzionalità lato client .